\documentclass[openany]{article}
\usepackage{lmodern}
\usepackage{amssymb,amsmath}
\usepackage{ifxetex,ifluatex}
\usepackage{fixltx2e} % provides \textsubscript
\ifnum 0\ifxetex 1\fi\ifluatex 1\fi=0 % if pdftex
  \usepackage[T1]{fontenc}
  \usepackage[utf8]{inputenc}
\else % if luatex or xelatex
  \ifxetex
    \usepackage{mathspec}
  \else
    \usepackage{fontspec}
  \fi
  \defaultfontfeatures{Ligatures=TeX,Scale=MatchLowercase}
\fi
% use upquote if available, for straight quotes in verbatim environments
\IfFileExists{upquote.sty}{\usepackage{upquote}}{}
% use microtype if available
\IfFileExists{microtype.sty}{%
\usepackage{microtype}
\UseMicrotypeSet[protrusion]{basicmath} % disable protrusion for tt fonts
}{}
\usepackage[margin=2cm]{geometry}
\usepackage{hyperref}
\PassOptionsToPackage{usenames,dvipsnames}{color} % color is loaded by hyperref
\hypersetup{unicode=true,
            pdftitle={IMOP: Workflowr Quick Start},
            pdfauthor={Ania Tassinari \& Caitlin Guccione},
            colorlinks=true,
            linkcolor=Maroon,
            citecolor=Blue,
            urlcolor=blue,
            breaklinks=true}
\urlstyle{same}  % don't use monospace font for urls
\usepackage{natbib}
\bibliographystyle{apalike}
\usepackage{color}
\usepackage{fancyvrb}
\newcommand{\VerbBar}{|}
\newcommand{\VERB}{\Verb[commandchars=\\\{\}]}
\DefineVerbatimEnvironment{Highlighting}{Verbatim}{commandchars=\\\{\}}
% Add ',fontsize=\small' for more characters per line
\usepackage{framed}
\definecolor{shadecolor}{RGB}{248,248,248}
\newenvironment{Shaded}{\begin{snugshade}}{\end{snugshade}}
\newcommand{\AlertTok}[1]{\textcolor[rgb]{0.94,0.16,0.16}{#1}}
\newcommand{\AnnotationTok}[1]{\textcolor[rgb]{0.56,0.35,0.01}{\textbf{\textit{#1}}}}
\newcommand{\AttributeTok}[1]{\textcolor[rgb]{0.77,0.63,0.00}{#1}}
\newcommand{\BaseNTok}[1]{\textcolor[rgb]{0.00,0.00,0.81}{#1}}
\newcommand{\BuiltInTok}[1]{#1}
\newcommand{\CharTok}[1]{\textcolor[rgb]{0.31,0.60,0.02}{#1}}
\newcommand{\CommentTok}[1]{\textcolor[rgb]{0.56,0.35,0.01}{\textit{#1}}}
\newcommand{\CommentVarTok}[1]{\textcolor[rgb]{0.56,0.35,0.01}{\textbf{\textit{#1}}}}
\newcommand{\ConstantTok}[1]{\textcolor[rgb]{0.00,0.00,0.00}{#1}}
\newcommand{\ControlFlowTok}[1]{\textcolor[rgb]{0.13,0.29,0.53}{\textbf{#1}}}
\newcommand{\DataTypeTok}[1]{\textcolor[rgb]{0.13,0.29,0.53}{#1}}
\newcommand{\DecValTok}[1]{\textcolor[rgb]{0.00,0.00,0.81}{#1}}
\newcommand{\DocumentationTok}[1]{\textcolor[rgb]{0.56,0.35,0.01}{\textbf{\textit{#1}}}}
\newcommand{\ErrorTok}[1]{\textcolor[rgb]{0.64,0.00,0.00}{\textbf{#1}}}
\newcommand{\ExtensionTok}[1]{#1}
\newcommand{\FloatTok}[1]{\textcolor[rgb]{0.00,0.00,0.81}{#1}}
\newcommand{\FunctionTok}[1]{\textcolor[rgb]{0.00,0.00,0.00}{#1}}
\newcommand{\ImportTok}[1]{#1}
\newcommand{\InformationTok}[1]{\textcolor[rgb]{0.56,0.35,0.01}{\textbf{\textit{#1}}}}
\newcommand{\KeywordTok}[1]{\textcolor[rgb]{0.13,0.29,0.53}{\textbf{#1}}}
\newcommand{\NormalTok}[1]{#1}
\newcommand{\OperatorTok}[1]{\textcolor[rgb]{0.81,0.36,0.00}{\textbf{#1}}}
\newcommand{\OtherTok}[1]{\textcolor[rgb]{0.56,0.35,0.01}{#1}}
\newcommand{\PreprocessorTok}[1]{\textcolor[rgb]{0.56,0.35,0.01}{\textit{#1}}}
\newcommand{\RegionMarkerTok}[1]{#1}
\newcommand{\SpecialCharTok}[1]{\textcolor[rgb]{0.00,0.00,0.00}{#1}}
\newcommand{\SpecialStringTok}[1]{\textcolor[rgb]{0.31,0.60,0.02}{#1}}
\newcommand{\StringTok}[1]{\textcolor[rgb]{0.31,0.60,0.02}{#1}}
\newcommand{\VariableTok}[1]{\textcolor[rgb]{0.00,0.00,0.00}{#1}}
\newcommand{\VerbatimStringTok}[1]{\textcolor[rgb]{0.31,0.60,0.02}{#1}}
\newcommand{\WarningTok}[1]{\textcolor[rgb]{0.56,0.35,0.01}{\textbf{\textit{#1}}}}
\usepackage{longtable,booktabs}
\usepackage{graphicx,grffile}
\makeatletter
\def\maxwidth{\ifdim\Gin@nat@width>\linewidth\linewidth\else\Gin@nat@width\fi}
\def\maxheight{\ifdim\Gin@nat@height>\textheight\textheight\else\Gin@nat@height\fi}
\makeatother
% Scale images if necessary, so that they will not overflow the page
% margins by default, and it is still possible to overwrite the defaults
% using explicit options in \includegraphics[width, height, ...]{}
\setkeys{Gin}{width=\maxwidth,height=\maxheight,keepaspectratio}
\IfFileExists{parskip.sty}{%
\usepackage{parskip}
}{% else
\setlength{\parindent}{0pt}
\setlength{\parskip}{6pt plus 2pt minus 1pt}
}
\setlength{\emergencystretch}{3em}  % prevent overfull lines
\providecommand{\tightlist}{%
  \setlength{\itemsep}{0pt}\setlength{\parskip}{0pt}}
\setcounter{secnumdepth}{5}
% Redefines (sub)paragraphs to behave more like sections
\ifx\paragraph\undefined\else
\let\oldparagraph\paragraph
\renewcommand{\paragraph}[1]{\oldparagraph{#1}\mbox{}}
\fi
\ifx\subparagraph\undefined\else
\let\oldsubparagraph\subparagraph
\renewcommand{\subparagraph}[1]{\oldsubparagraph{#1}\mbox{}}
\fi

%%% Use protect on footnotes to avoid problems with footnotes in titles
\let\rmarkdownfootnote\footnote%
\def\footnote{\protect\rmarkdownfootnote}

%%% Change title format to be more compact
\usepackage{titling}

% Create subtitle command for use in maketitle
\providecommand{\subtitle}[1]{
  \posttitle{
    \begin{center}\large#1\end{center}
    }
}

\setlength{\droptitle}{-2em}

  \title{IMOP: Workflowr Quick Start}
    \pretitle{\vspace{\droptitle}\centering\huge}
  \posttitle{\par}
    \author{Ania Tassinari \& Caitlin Guccione}
    \preauthor{\centering\large\emph}
  \postauthor{\par}
      \predate{\centering\large\emph}
  \postdate{\par}
    \date{Summer 2019}


\begin{document}
\maketitle

\hypertarget{set-up---once-per-machine}{%
\section{Set Up - Once Per Machine}\label{set-up---once-per-machine}}

It is ideal to do this setup in RStudio, but any platform of R will work:

\begin{Shaded}
\begin{Highlighting}[]
\KeywordTok{install.packages}\NormalTok{(}\StringTok{"workflowr"}\NormalTok{)}
\KeywordTok{library}\NormalTok{(}\StringTok{"workflowr"}\NormalTok{)}
\KeywordTok{wflow_git_config}\NormalTok{(}\DataTypeTok{user.name =} \StringTok{"First Last"}\NormalTok{, }\DataTypeTok{user.email =} \StringTok{"first.last@agios.com"}\NormalTok{)}
\end{Highlighting}
\end{Shaded}

\hypertarget{create-project}{%
\section{Create Project}\label{create-project}}

\hypertarget{r-prep}{%
\subsection{R Prep}\label{r-prep}}

\begin{Shaded}
\begin{Highlighting}[]
\KeywordTok{library}\NormalTok{(}\StringTok{"workflowr"}\NormalTok{) }\CommentTok{#Don't have to repeat if completed above}
\KeywordTok{wflow_start}\NormalTok{(}\StringTok{"/path/PROJECT"}\NormalTok{) }\CommentTok{#Starts a workflowr project by creating folders & R project}
\end{Highlighting}
\end{Shaded}

\hypertarget{gitlab-repo-connection}{%
\subsection{GitLab Repo Connection}\label{gitlab-repo-connection}}

Go to your Agios GitLab and do the following:

\begin{itemize}
\tightlist
\item
  Create a project in GitLab with the same name as the project in RStudio
\item
  Go to your PROJECT directory in your \texttt{terminal} and paste the \texttt{Push\ an\ Existing\ Git\ Repository}
\end{itemize}

\hypertarget{complete-connection-in-r}{%
\subsection{Complete connection in R}\label{complete-connection-in-r}}

\begin{Shaded}
\begin{Highlighting}[]
\KeywordTok{wflow_build}\NormalTok{() }\CommentTok{# Builds the Rmd files which builds the site allowing you to preview changes}
\KeywordTok{wflow_publish}\NormalTok{(}\KeywordTok{c}\NormalTok{(}\StringTok{"analysis/*.Rmd"}\NormalTok{), }\StringTok{"Publish the initial files for PROJECT"}\NormalTok{) }\CommentTok{#Updates HTML files}
\KeywordTok{wflow_use_gitlab}\NormalTok{(}\DataTypeTok{username =} \StringTok{"first.last"}\NormalTok{, }\DataTypeTok{repository =} \StringTok{"PROJECT"}\NormalTok{, }\DataTypeTok{domain =} \StringTok{"git.agios.local"}\NormalTok{)}
\end{Highlighting}
\end{Shaded}

\hypertarget{creating-a-new-file-running-analysis}{%
\section{Creating a New File \& Running Analysis}\label{creating-a-new-file-running-analysis}}

\begin{Shaded}
\begin{Highlighting}[]
\KeywordTok{wflow_open}\NormalTok{(}\StringTok{"analysis/NEW.Rmd"}\NormalTok{) }\CommentTok{#Creates and opens a new file}
\KeywordTok{wflow_build}\NormalTok{() }\CommentTok{# If you are in RStudio you can also use the Knitr button here}
\KeywordTok{wflow_publish}\NormalTok{(}\KeywordTok{c}\NormalTok{(}\StringTok{"analysis/NEW.Rmd"}\NormalTok{), }\StringTok{"Publish the file NEW"}\NormalTok{) }\CommentTok{#Updates HTML files}
\end{Highlighting}
\end{Shaded}

Make sure you are in the correct project in your \texttt{terminal}, then run:

\begin{Shaded}
\begin{Highlighting}[]
\NormalTok{git push }\CommentTok{#Pushes your work to Git saving a backup and allowing others to view your updated page}
\end{Highlighting}
\end{Shaded}

Repeat this process anytime you want to update your page.

\bibliography{book.bib,packages.bib}


\end{document}
